\documentclass[twocolumn,twoside,11pt]{article}

\usepackage[portuguese]{babel}  % portuguese
\usepackage{graphicx}           % images: .png or .pdf w/ pdflatex; .eps w/ latex

\usepackage{lipsum}             % generate dummy text throughout this template

%% For iso-8859-1 (latin1), comment next line and uncomment the second line
\usepackage[utf8]{inputenc}
%\usepackage[latin1]{inputenc}

\usepackage[T1]{fontenc}        % T1 fonts
\usepackage{lmodern}            % fonts
\usepackage[sc]{mathpazo}       % Use the Palatino font
\linespread{1.05}               % Line spacing - Palatino needs more space between lines
\usepackage{microtype}          % Slightly tweak font spacing for aesthetics
\usepackage{url}                % urls
\usepackage[hang, small, labelfont=bf,up,textfont=it,up]{caption} % Custom captions under/above floats in tables or figures
\usepackage{booktabs}           % Horizontal rules in tables
\usepackage{float}              % Required for tables and figures in the multi-column environment - they need to be placed in specific locations with the [H] (e.g. \begin{table}[H])
\usepackage{paralist}           % Used for the compactitem environment which makes bullet points with less space between them

% geometry package
\usepackage[outer=20mm,inner=20mm,vmargin=15mm,includehead,includefoot,headheight=15pt]{geometry}
%% space between columns
\columnsep 10mm

\usepackage{abstract}           % Allows abstract customization
\renewcommand{\abstractnamefont}{\normalfont\bfseries} % Set the "Abstract" text to bold
\renewcommand{\abstracttextfont}{\normalfont\small\itshape} % Set the abstract itself to small italic text

% \usepackage{titlesec}           % Allows customization of titles
% \renewcommand\thesection{\Roman{section}} % Roman numerals for the sections
% \renewcommand\thesubsection{\Roman{subsection}} % Roman numerals for subsections
% \titleformat{\section}[block]{\large\scshape\centering}{\thesection.}{1em}{} % Change the look of the section titles
% \titleformat{\subsection}[block]{\large}{\thesubsection.}{1em}{} % Change the look of the section titles

\usepackage[pdftex]{hyperref}
\hypersetup{%
    colorlinks = true,           % false: boxed links; true: colored links
    pdffitwindow = false,        % page fit to window when opened
    pdfpagemode = UseNone,       % do not show bookmarks
    pdfpagelayout = SinglePage,  % displays a single page
    pdfpagetransition = Replace, % page transition
    linkcolor=blue,              % hyperlink colors
    urlcolor=blue,
    citecolor=blue,
    anchorcolor=green
}

\usepackage{indentfirst}         % indent also 1st paragraph

\usepackage{fancyhdr}            % Headers and footers
\pagestyle{fancy}                % pages have headers and footers
\fancyhead{}                     % Blank out the default header
\fancyfoot{}                     % Blank out the default footer
% \fancyhead[LO,RE]{Exemplo de artigo em \LaTeX} % Custom header text
% \fancyhead[RO,LE]{\thepage}      % Custom header text
% \fancyfoot[RO,LE]{Grupo xx, \today} % Custom footer text
\renewcommand{\headrulewidth}{0.4pt}
\renewcommand{\footrulewidth}{0.4pt}

%\hyphenation{}                  % explicit hyphenation

%---------------------------------------------------------------------------------------
%	macro definitions
%---------------------------------------------------------------------------------------

% entities
\newcommand{\class}[1]{{\normalfont\slshape #1\/}}
\newcommand{\svg}{\class{SVG}}
\newcommand{\scadadms}{\class{SCADA/DMS}}

%----------------------------------------------------------------------------------------
%	TITLE SECTION
%----------------------------------------------------------------------------------------

\title{\vspace{-15mm}\fontsize{24pt}{10pt}\selectfont\textbf{
  Spotter: Localização interior com \emph{QRCodes} usando dispositivos móveis
}}

\author{José Bateira\\
\small \texttt{ei10133@fe.up.pt}\\
\small Faculdade de Engenharia da Universidade de Porto
\and
Rui Rodrigues\\
\small \texttt{rui.rodrigues@fe.up.pt} \\
\small Faculdade de Engenharia da Universidade de Porto
\vspace{-5mm}
}

\date{\today}

%----------------------------------------------------------------------------------------

\begin{document}

\maketitle
\thispagestyle{plain}            % no headers in the first page

%----------------------------------------------------------------------------------------
%	ABSTRACT
%----------------------------------------------------------------------------------------

\begin{abstract}
Localização interior de edifícios e infraestruturas com smartphones/tablets é um tema bastante falado que permite com algum tipo de tecnologia (\emph{wifi}, \emph{bluetooth}) localizar um utilizador no mapa do edifício onde se encontra.
% As suas aplicações são diversas, como por exemplo, visitas guiadas a museus, mapas de aeroportos, hospitais e centros comerciais.
% A maioria das soluções usam wifi, bluetooh, RFID cards, Bússola, giroscópio e acelerómetro. [?]
% É de notar que a quantidade de recursos do dispositivo móvel que estão a ser usados são bastantes, o que pode fazer com que se gaste muita bateria.
% O GPS não é uma solução viável pois não funciona em ambientes cobertos.

A solução proposta foca-se em dar informação da localização atual \emph{on-demand} e não em \emph{real-time}, sem necessitar de tecnologias \emph{wireless}. Quando um utilizador lê um \emph{QRCode} afixado num edifício é redirecionado para um website (adaptado para mobile) que mostra o mapa do edifício com um marker que indica a posição do utilizador.

% Os resultados desta solução apontam para uma satisfação na rapidez com que se consegue saber a posição, pois tudo fica dependente da velocidade de conexão do dispositivo do utilizador à internet.
\end{abstract}

%----------------------------------------------------------------------------------------
%	ARTICLE CONTENTS
%----------------------------------------------------------------------------------------

\section{Introdução}\label{sec:intro}

  Desde que os dispositivos móveis passaram a suportar conexões \emph{wireless}, muitas soluções para localização interior surgiram.
  Usando wifi, bluetooth e até RFID, a falta de precisão da posição e o consumo de bateria excessivo são alguns dos problemas que não tornam estas soluções viáveis.
  No entanto, estas têm em foco um ponto bastante importante: localização em tempo real.

  A solução proposta aborda o problema com outro paradigma: localização por pedido (\emph{on-demand, non-real time}).
  Aquando da leitura de um QRCode devidamente afixado num ponto de um edifício, o utilizador é redirecionado para um website (adaptado para visualização \emph{mobile}) que mostra a parte da planta do mapa do edifício onde o utilizador se encontra.
  Um apontador deve indicar a posição do utilizador no mapa.


  % section introducao (end)

\section{Estado da Arte} % (fold)
\label{sec:state_of_the_art}

\lipsum[3]

\subsection{Projeto 1} % (fold)
\label{sub:projeto_1}
\lipsum[2]
% subsection projeto_1 (end)


\subsection{Projeto 2} % (fold)
\label{sub:projeto_2}
\lipsum[7]

% subsection projeto_2 (end)

% section state_of_the_art (end)


\section{Solução Proposta} % (fold)
\label{sec:solucao}

\lipsum[8]

\subsection{Resultados} % (fold)
\label{sub:resultados}
\lipsum[3]
% subsection resultados (end)

% section solucao (end)


\section{Conclusões}\label{sec:conclusions}

\lipsum[8]

%----------------------------------------------------------------------------------------
%	REFERENCE LIST
%----------------------------------------------------------------------------------------

%% auto bibliographic list 
\renewcommand{\bibname}{Referências}
% uses bibtex file
%\bibliographystyle{alpha-pt}
%\bibliographystyle{alpha}
\bibliographystyle{unsrt-pt}
%\bibliographystyle{unsrt}
\bibliography{bib/myrefs}

%----------------------------------------------------------------------------------------

\end{document}


